%% Basierend auf einer TeXnicCenter-Vorlage von Tino Weinkauf.
%%%%%%%%%%%%%%%%%%%%%%%%%%%%%%%%%%%%%%%%%%%%%%%%%%%%%%%%%%%%%%

%%%%%%%%%%%%%%%%%%%%%%%%%%%%%%%%%%%%%%%%%%%%%%%%%%%%%%%%%%%%%
%% OPTIONEN
%%%%%%%%%%%%%%%%%%%%%%%%%%%%%%%%%%%%%%%%%%%%%%%%%%%%%%%%%%%%%
%%
%% ACHTUNG: Sie benötigen ein Hauptdokument, um diese Datei
%%          benutzen zu können. Verwenden Sie im Hauptdokument
%%          den Befehl "\input{dateiname}", um diese
%%          Datei einzubinden.
%%

%%%%%%%%%%%%%%%%%%%%%%%%%%%%%%%%%%%%%%%%%%%%%%%%%%%%%%%%%%%%%
%% OPTIONEN FÜR ABSTÄNDE
%%%%%%%%%%%%%%%%%%%%%%%%%%%%%%%%%%%%%%%%%%%%%%%%%%%%%%%%%%%%%

%%Abstand zwischen den Absätzen: halbe Höhe vom kleinen x
\setlength{\parskip}{0.5ex}

%%Einzug am Anfang eines Absatzes: auf Null setzen
%\setlength{\parindent}{0ex}

%%Zeilenabstand: 1.5 fach
%% ==> Erwägen Sie, anstelle dieses Kommandos das Paket 'setspace' zu verwenden.
%\linespread{1.5}

%%Abstände zu Seitenränder definieren
\usepackage[left=25mm, right=25mm, top=25mm, bottom=25mm]{geometry}

%%%%%%%%%%%%%%%%%%%%%%%%%%%%%%%%%%%%%%%%%%%%%%%%%%%%%%%%%%%%%%%%%%%%%%%%%%%%%%%
%%Seitenlayout definieren
\geometry{a4paper,
  left  = 25mm,
  right = 25mm,
  top   = 25mm,
  bottom= 25mm,
}
\geometry{a4paper,
	tmargin=2.5cm,
	bmargin=2.5cm,
	lmargin=3.5cm,
	rmargin=3.0cm
}

%%%%%%%%%%%%%%%%%%%%%%%%%%%%%%%%%%%%%%%%%%%%%%%%%%%%%%%%%%%%%
%% OPTIONEN FÜR KOPF- UND FUSSZEILEN
%%%%%%%%%%%%%%%%%%%%%%%%%%%%%%%%%%%%%%%%%%%%%%%%%%%%%%%%%%%%%
%%Beispiel für recht nette Kopf- und Fußzeilen
%% ==> Nutzen Sie '\usepackage{scrlayer-scrpage}'
%% ==> im Hauptdokument, um diese zu benutzen.
\usepackage{scrlayer-scrpage}
%\pagestyle{fancy}
%\renewcommand{\chaptermark}[1]{\markboth{#1}{}}
\renewcommand{\sectionmark}[1]{\markright{\thesection\ #1}}
%\fancyhf{}
%\fancyhead[LE,RO]{\thepage}
%\fancyhead[LO]{\rightmark}
%\fancyhead[RE]{\leftmark}
%\fancypagestyle{plain}{%
%    \fancyhead{}
%    \renewcommand{\headrulewidth}{0pt}
%}


%Quelle: https://esc-now.de/_/latex-individuelle-kopf--und-fusszeilen-update/

%\pagestyle{scrheadings}		
%\clearpairofpagestyles	%leert alle sechs Platzhalter	
%\ihead{Kopfzeile innen}	
%\chead{Kopfzeile Mitte}	
%\ohead{3. Übung}				%aktuelle Kolumnentitel (z.B. Chapter, Section oder Subsection) wird eingefügt
%\ifoot{Fußzeile innen}	
%\cfoot{Fußzeile Mitte}	
%\ofoot{\pagemark}			%Seitenzahl wird in der Fußzeile außen ausgegeben
%\automark[subsection]{section}

%% Fuß- und Kopfzeile bei Kapitelseiten:
%\renewcommand{\chapterpagestyle}{scrheadings}

%%%%%%%%%%%%%%%%%%%%%%%%%%%%%%%%%%%%%%%%%%%%%%%%%%%%%%%%%%%%%%%%%%%%%%%%%%%%%%%
%% Quellenformatierung für Tabellen und Grafiken
\newcommand*{\quelle}[1]{
	\centerline{\footnotesize(\underline{Quelle:} #1)}
}

%%%%%%%%%%%%%%%%%%%%%%%%%%%%%%%%%%%%%%%%%%%%%%%%%%%%%%%%%%%%%
%% OPTIONEN FÜR Quellcode
%%%%%%%%%%%%%%%%%%%%%%%%%%%%%%%%%%%%%%%%%%%%%%%%%%%%%%%%%%%%%
\usepackage{color}
    \definecolor{lightgray}{rgb}{0.95, 0.95, 0.95}
    \definecolor{darkgray}{rgb}{0.4, 0.4, 0.4}
    \definecolor{purple}{rgb}{0.65, 0.12, 0.82}
    \definecolor{ocherCode}{rgb}{1, 0.5, 0} % #FF7F00 -> rgb(239, 169, 0)
    \definecolor{blueCode}{rgb}{0, 0, 0.93} % #0000EE -> rgb(0, 0, 238)
    \definecolor{greenCode}{rgb}{0, 0.6, 0} % #009900 -> rgb(0, 153, 0) 
\usepackage{upquote}
\usepackage{scrhack}
\usepackage{listings}
%\makeatletter
\lstset{%
   % Basic design
   backgroundcolor=\color{lightgray},
   basicstyle={\small\ttfamily},   
   frame=l,
   % Line numbers
   xleftmargin={0.75cm},
   numbers=left,
	numberstyle=\tiny,
   stepnumber=1,
   firstnumber=1,
   numberfirstline=true,
   % Code design
   identifierstyle=\color{black},
   keywordstyle=\color{blue}\bfseries,
   ndkeywordstyle=\color{greenCode}\bfseries,
   stringstyle=\color{ocherCode}\ttfamily,
   commentstyle=\color{darkgray}\ttfamily,
   % Code
   %language={HTML5},
   tabsize=2,
   showtabs=false,
   showspaces=false,
   showstringspaces=false,
   extendedchars=true,
   breaklines=true,
	breakautoindent=true,
	%breakatwhitespace=true,
}
% enable german umlauts in listings
\lstset{
  literate={ö}{{\"o}}1
           {Ö}{{\"O}}1
           {ä}{{\"a}}1
           {Ä}{{\"A}}1
           {ü}{{\"u}}1
           {Ü}{{\"U}}1
           {ß}{{\ss}}1
}
\lstdefinelanguage{HTML5}{
    sensitive=true,
    keywords={%
    % JavaScript
    typeof, new, true, false, catch, function, return, null, catch, switch, var, if, in, while, do, else, case, break,
    % HTML
    html, title, meta, style, head, body, script, canvas,
    % CSS
    border:, transform:, -moz-transform:, transition-duration:, transition-property:,
    transition-timing-function:
    },
    % http://texblog.org/tag/otherkeywords/
    otherkeywords={<, >, \/},   
    ndkeywords={class, export, boolean, throw, implements, import, this},   
    comment=[l]{//},
    % morecomment=[s][keywordstyle]{<}{>},  
    morecomment=[s]{/*}{*/},
    morecomment=[s]{<!}{>},
    morestring=[b]',
    morestring=[b]",    
    alsoletter={-},
    alsodigit={:}
}
%%%%%%%%%%%%%%%%%%%%%%%%%%%%%%%%%%%%%%%%%%%%%%%%%%%%%%%%%%%%%%%%%%%%%%%%%%%%%%%
%%%%%%%%%%%%%%%%%%%%%%%%%%%%%%%%%%%%%%%%%%%%%%%%%%%%%%%%%%%%%%%%%%%%%%%%%%%%%%%
%%%% Im Text Tab. und Abb. und Lsiting Captions werden Fett ausgegeben und anschließendem Doppelpunkt :
\usepackage{caption}
\DeclareCaptionFormat{underline}{\underline{#1:}#2#3}
\captionsetup{format=underline, labelsep=space,labelfont=bf}

\newcaptionname{ngerman}{\listoflolentryname}{\lstlistingname}

\newcommand\entrywithprefixformat[1]{%
  \def\autodot{:}%
  \bfseries\underline{#1}%
}

\DeclareTOCStyleEntry[
  entrynumberformat=\entrywithprefixformat,
  dynnumwidth
]{default}{figure}

\DeclareTOCStyleEntry[
  entrynumberformat=\entrywithprefixformat,
  dynnumwidth
]{default}{table}

\DeclareTOCStyleEntry[
  level=1,
  indent=1.5em,
  numwidth=2.3em,
  entrynumberformat=\entrywithprefixformat,
  dynnumwidth
]{default}{lstlisting}
