%% Basierend auf einer TeXnicCenter-Vorlage von Mark Müller
%%%%%%%%%%%%%%%%%%%%%%%%%%%%%%%%%%%%%%%%%%%%%%%%%%%%%%%%%%%%%%%%%%%%%%%

% Wählen Sie die Optionen aus, indem Sie % vor der Option entfernen  
% Dokumentation des KOMA-Script-Packets: scrguide

%%%%%%%%%%%%%%%%%%%%%%%%%%%%%%%%%%%%%%%%%%%%%%%%%%%%%%%%%%%%%%%%%%%%%%%
%% Optionen zum Layout des Artikels                                  %%
%%%%%%%%%%%%%%%%%%%%%%%%%%%%%%%%%%%%%%%%%%%%%%%%%%%%%%%%%%%%%%%%%%%%%%%
\documentclass[a4paper,%
%a5paper,						% alle weiteren Papierformat einstellbar
%landscape,						% Querformat
%10pt,							% Schriftgröße (12pt, 11pt (Standard))
%BCOR1cm,						% Bindekorrektur, bspw. 1 cm
%DIVcalc,						% führt die Satzspiegelberechnung neu aus
%											  s. scrguide 2.4
twoside,							% Doppelseiten
%twocolumn,						% zweispaltiger Satz
%halfparskip*,					% Absatzformatierung s. scrguide 3.1
headsepline=true,				% Trennline zum Seitenkopf	
footsepline=true,				% Trennline zum Seitenfuß
%titlepage,						% Titelei auf eigener Seite
%normalheadings,				% Überschriften etwas kleiner (smallheadings)
headings=normal,
%idxtotoc,						% Index im Inhaltsverzeichnis
%liststotoc,					% Abb.- und Tab.verzeichnis im Inhalt
%bibtotoc,						% Literaturverzeichnis im Inhalt2
%index=totoc, 					% Index im Inhaltsverzeichnis aufführen
listof=totoc,					% Verzeichnisse im Inhaltsverzeichnis aufführen
bibliography=totoc,			% Literaturverzeichnis im Inhaltsverzeichnis aufführen
listof=entryprefix,
%numbers=noenddot,			%Keine Punkte am Ende des TOC
%toc=flat,						%Flache TOC
%abstracton,					% Überschrift über der Zusammenfassung an	
%leqno,   						% Nummerierung von Gleichungen links
%fleqn,								% Ausgabe von Gleichungen linksbündig
%draft								% überlangen Zeilen in Ausgabe gekennzeichnet
%chapterprefix=on,
%, draft
]
{scrartcl}

%\pagestyle{empty}		% keine Kopf und Fußzeile (k. Seitenzahl)
%\pagestyle{headings}	% lebender Kolumnentitel  

%% Deutsche Anpassungen %%%%%%%%%%%%%%%%%%%%%%%%%%%%%%%%%%%%%
\usepackage[ngerman]{babel}
\usepackage[T1]{fontenc}
\usepackage[utf8]{inputenc}

\usepackage{lmodern} %Type1-Schriftart für nicht-englische Texte


%% Packages für Grafiken & Abbildungen %%%%%%%%%%%%%%%%%%%%%%
%\usepackage{graphicx} %%Zum Laden von Grafiken
%\usepackage{subfig} %%Teilabbildungen in einer Abbildung
%\usepackage{pst-all} %%PSTricks - nicht verwendbar mit pdfLaTeX

%% Beachten Sie:
%% Die Einbindung einer Grafik erfolgt mit \includegraphics{Dateiname}
%% bzw. über den Dialog im Einfügen-Menü.
%% 
%% Im Modus "LaTeX => PDF" können Sie u.a. folgende Grafikformate verwenden:
%%   .jpg  .png  .pdf  .mps
%% 
%% In den Modi "LaTeX => DVI", "LaTeX => PS" und "LaTeX => PS => PDF"
%% können Sie u.a. folgende Grafikformate verwenden:
%%   .eps  .ps  .bmp  .pict  .pntg


%% Bibliographiestil %%%%%%%%%%%%%%%%%%%%%%%%%%%%%%%%%%%%%%%%%%%%%%%%%%
\usepackage[
	backend = biber
	, natbib	= true
	, bibwarn = true
	, bibencoding = utf8
	, style = alphabetic
	, sorting = ydnt,
	, maxbibnames = 99
	, date = long
	, abbreviate = false
	, language = ngerman
	%sorting = ynt
]{biblatex}
%\usepackage[
%	backend = biber,
%	natbib  = true,
%	bibwarn = true,
%   style	  = alphabetic,
%	sorting = ydnt,
%   %isbn	= true,		%default ist true
%   %url	= true,		%default ist true
%   %doi	= true,		%default ist true
%]{biblatex}
\addbibresource{Literatur.bib}
\usepackage{csquotes}


%% Basierend auf einer TeXnicCenter-Vorlage von Tino Weinkauf.
%%%%%%%%%%%%%%%%%%%%%%%%%%%%%%%%%%%%%%%%%%%%%%%%%%%%%%%%%%%%%%

%%%%%%%%%%%%%%%%%%%%%%%%%%%%%%%%%%%%%%%%%%%%%%%%%%%%%%%%%%%%%
%% OPTIONEN
%%%%%%%%%%%%%%%%%%%%%%%%%%%%%%%%%%%%%%%%%%%%%%%%%%%%%%%%%%%%%
%%
%% ACHTUNG: Sie benötigen ein Hauptdokument, um diese Datei
%%          benutzen zu können. Verwenden Sie im Hauptdokument
%%          den Befehl "\input{dateiname}", um diese
%%          Datei einzubinden.
%%

%%%%%%%%%%%%%%%%%%%%%%%%%%%%%%%%%%%%%%%%%%%%%%%%%%%%%%%%%%%%%
%% OPTIONEN FÜR ABSTÄNDE
%%%%%%%%%%%%%%%%%%%%%%%%%%%%%%%%%%%%%%%%%%%%%%%%%%%%%%%%%%%%%

%%Abstand zwischen den Absätzen: halbe Höhe vom kleinen x
\setlength{\parskip}{0.5ex}

%%Einzug am Anfang eines Absatzes: auf Null setzen
%\setlength{\parindent}{0ex}

%%Zeilenabstand: 1.5 fach
%% ==> Erwägen Sie, anstelle dieses Kommandos das Paket 'setspace' zu verwenden.
%\linespread{1.5}

%%Abstände zu Seitenränder definieren
\usepackage[left=25mm, right=25mm, top=25mm, bottom=25mm]{geometry}

%%%%%%%%%%%%%%%%%%%%%%%%%%%%%%%%%%%%%%%%%%%%%%%%%%%%%%%%%%%%%
%% OPTIONEN FÜR KOPF- UND FUSSZEILEN
%%%%%%%%%%%%%%%%%%%%%%%%%%%%%%%%%%%%%%%%%%%%%%%%%%%%%%%%%%%%%
%%Beispiel für recht nette Kopf- und Fußzeilen
%% ==> Nutzen Sie '\usepackage{scrlayer-scrpage}'
%% ==> im Hauptdokument, um diese zu benutzen.
%\pagestyle{fancy}
%\renewcommand{\chaptermark}[1]{\markboth{#1}{}}
%\renewcommand{\sectionmark}[1]{\markright{\thesection\ #1}}
%\fancyhf{}
%\fancyhead[LE,RO]{\thepage}
%\fancyhead[LO]{\rightmark}
%\fancyhead[RE]{\leftmark}
%\fancypagestyle{plain}{%
%    \fancyhead{}
%    \renewcommand{\headrulewidth}{0pt}
%}
%\ihead{Kopfzeile innen}	
%\chead{Kopfzeile Mitte}	
%\ohead{Kopfzeile außen}	
%\ifoot{Fußzeile innen}	
%\cfoot{Fußzeile Mitte}	
%\ofoot{Fußzeile außen}

%Quelle: https://esc-now.de/_/latex-individuelle-kopf--und-fusszeilen-update/
\usepackage{scrlayer-scrpage}
\pagestyle{scrheadings}
%\clearpairofpagestyle			%leert alle sechs Platzhalter??
\clearpairofpagestyles
\ohead{3. Übung}			%aktuelle Kolumnentitel (z.B. Chapter, Section oder Subsection) wird eingefügt
\ofoot{\pagemark}			%Seitenzahl wird in der Fußzeile außen ausgegeben
%\ofoot{\pagemark}			%Seitenzahl wird in der Fußzeile außen ausgegeben
%\automark[subsection]{section}

%%%%%%%%%%%%%%%%%%%%%%%%%%%%%%%%%%%%%%%%%%%%%%%%%%%%%%%%%%%%%
%% OPTIONEN FÜR Quellcode
%%%%%%%%%%%%%%%%%%%%%%%%%%%%%%%%%%%%%%%%%%%%%%%%%%%%%%%%%%%%%
\usepackage{color}
\definecolor{lightgray}{rgb}{0.95, 0.95, 0.95}
\definecolor{darkgray}{rgb}{0.4, 0.4, 0.4}
%\definecolor{purple}{rgb}{0.65, 0.12, 0.82}
\definecolor{editorGray}{rgb}{0.95, 0.95, 0.95}
\definecolor{editorOcher}{rgb}{1, 0.5, 0} % #FF7F00 -> rgb(239, 169, 0)
\definecolor{editorGreen}{rgb}{0, 0.5, 0} % #007C00 -> rgb(0, 124, 0)
\definecolor{orange}{rgb}{1,0.45,0.13}		
\definecolor{olive}{rgb}{0.17,0.59,0.20}
\definecolor{brown}{rgb}{0.69,0.31,0.31}
\definecolor{purple}{rgb}{0.38,0.18,0.81}
\definecolor{lightblue}{rgb}{0.1,0.57,0.7}
\definecolor{lightred}{rgb}{1,0.4,0.5}
\usepackage{upquote}
\usepackage{listings}
% CSS
\lstdefinelanguage{CSS}{
  keywords={color,background-image:,margin,padding,font,weight,display,position,top,left,right,bottom,list,style,border,size,white,space,min,width, transition:, transform:, transition-property, transition-duration, transition-timing-function},	
  sensitive=true,
  morecomment=[l]{//},
  morecomment=[s]{/*}{*/},
  morestring=[b]',
  morestring=[b]",
  alsoletter={:},
  alsodigit={-}
}

% JavaScript
\lstdefinelanguage{JavaScript}{
  morekeywords={typeof, new, true, false, catch, function, return, null, catch, switch, var, if, in, while, do, else, case, break},
  morecomment=[s]{/*}{*/},
  morecomment=[l]//,
  morestring=[b]",
  morestring=[b]'
}

\lstdefinelanguage{HTML5}{
  language=html,
  sensitive=true,	
  alsoletter={<>=-},	
  morecomment=[s]{<!-}{-->},
  tag=[s],
  otherkeywords={
  % General
  >,
  % Standard tags
	<!DOCTYPE,
  </html, <html, <head, <title, </title, <style, </style, <link, </head, <meta, />,
	% body
	</body, <body,
	% Divs
	</div, <div, </div>, 
	% Paragraphs
	</p, <p, </p>,
	% scripts
	</script, <script,
  % More tags...
  <canvas, /canvas>, <svg, <rect, <animateTransform, </rect>, </svg>, <video, <source, <iframe, </iframe>, </video>, <image, </image>, <header, </header, <article, </article
  },
  ndkeywords={
  % General
  =,
  % HTML attributes
  charset=, src=, id=, width=, height=, style=, type=, rel=, href=,
  % SVG attributes
  fill=, attributeName=, begin=, dur=, from=, to=, poster=, controls=, x=, y=, repeatCount=, xlink:href=,
  % properties
  margin:, padding:, background-image:, border:, top:, left:, position:, width:, height:, margin-top:, margin-bottom:, font-size:, line-height:,
	% CSS3 properties
  transform:, -moz-transform:, -webkit-transform:,
  animation:, -webkit-animation:,
  transition:,  transition-duration:, transition-property:, transition-timing-function:,
% German umlauts
  literate=%
  {Ö}{{\"O}}
  {Ä}{{\"A}}
  {Ü}{{\"U}}
  {ß}{{\ss}}
  {ü}{{\"u}}
  {ä}{{\"a}}
  {ö}{{\"o}}
  }
}

\lstdefinestyle{htmlcssjs} {%
  % General design
%  backgroundcolor=\color{editorGray},
  basicstyle={\footnotesize\ttfamily},   
  frame=b,
  % line-numbers
  xleftmargin={0.75cm},
  numbers=left,
  stepnumber=1,
  firstnumber=1,
  numberfirstline=true,	
  % Code design
  identifierstyle=\color{black},
  keywordstyle=\color{blue}\bfseries,
  ndkeywordstyle=\color{editorGreen}\bfseries,
  stringstyle=\color{editorOcher}\ttfamily,
  commentstyle=\color{brown}\ttfamily,
  % Code
  language=HTML5,
  alsolanguage=JavaScript,
  alsodigit={.:;},	
  tabsize=2,
  showtabs=false,
  showspaces=false,
  showstringspaces=false,
  extendedchars=true,
  breaklines=true,
  inputcoding=utf8,
% German umlauts
  literate=%
  {Ö}{{\"O}}
  {Ä}{{\"A}}
  {Ü}{{\"U}}
  {ß}{{\ss}}
  {ü}{{\"u}}
  {ä}{{\"a}}
  {ö}{{\"o}}
}
%%%%%%%%%%%%%%%%%%%%%%%%%%%%%%%%%%%%%%%%%%%%%%%%%%%%%%%%%%%%%%
%% PDF-Informationen
%%%%%%%%%%%%%%%%%%%%%%%%%%%%%%%%%%%%%%%%%%%%%%%%%%%%%%%%%%%%%
%%
%% ACHTUNG: Sie benötigen ein Hauptdokument, um diese Datei
%%          benutzen zu können. Verwenden Sie im Hauptdokument
%%          den Befehl "\input{dateiname}", um diese
%%          Datei einzubinden.
%%
\usepackage[pdftex
						, pdfauthor={Dogan Alkan}
						, pdftitle={Entwicklung einer Softarelösung zum Energiemonitoring}
						, pdfsubject={Masterarbeit im Studiengang Medieninformatik Master of Science}
						, pdfkeywords={Dogan Alkan}{Masterarbeit}{Raspberry Pi}{UART}
						, pdfproducer={Dogan Alkan}
						, pdfcreator={pdflatex}
						%, colorlinks=true
						%, linkcolor=black
						%, citecolor=black
						%, filecolor=black
						, hidelinks
						%, pagecolor=black
						, urlcolor=black
						, bookmarks=true
						, bookmarksopen=true
						]{hyperref}
\begin{document}

\renewcommand*{\thesection}{\arabic{section}}
\renewcommand*{\thesubsection}{\alph{subsection}}
%\renewcommand*{\p@subsection}{\thesection}
\renewcommand*{\thesubsubsection}{\arabic{subsubsection}}
%\renewcommand*{\p@subsubsection}{\p@subsection\thesubsection)

%\pagestyle{scrheadings} %%Keine Kopf-/Fusszeilen auf den ersten Seiten.


%%%%%%%%%%%%%%%%%%%%%%%%%%%%%%%%%%%%%%%%%%%%%%%%%%%%%%%%%%%%%%%%%%%%%%%
%% Ihr Artikel                                                       %%
%%%%%%%%%%%%%%%%%%%%%%%%%%%%%%%%%%%%%%%%%%%%%%%%%%%%%%%%%%%%%%%%%%%%%%%

%% eigene Titelseitengestaltung %%%%%%%%%%%%%%%%%%%%%%%%%%%%%%%%%%%%%%%    
%\begin{titlepage}
%Einsetzen der TXC Vorlage "Deckblatt" möglich
%\end{titlepage}

%% Angaben zur Standardformatierung des Titels %%%%%%%%%%%%%%%%%%%%%%%%
%\titlehead{Titelkopf }
\subject{\noindent\hrulefill\\ Multimediale Informationsverarbeitung}
\title{3. Übung: Lösungen}
\author{Dogan Alkan, s835118}
%\and{Der Name des Co-Autoren}
%\thanks{Fußnote}			% entspr. \footnote im Fließtext
\date{\noindent\hrulefill}	\noindent						% falls anderes, als das aktuelle gewünscht
%\publishers{Herausgeber}
%% Widmungsseite %%%%%%%%%%%%%%%%%%%%%%%%%%%%%%%%%%%%%%%%%%%%%%%%%%%%%%
%\dedication{Widmung}
\maketitle 						% Titelei wird erzeugt

%% Zusammenfassung nach Titel, vor Inhaltsverzeichnis %%%%%%%%%%%%%%%%%
%\begin{abstract}
% Für eine kurze Zusammenfassung des folgenden Artikels.
% Für die Überschrift s. \documentclass[abstracton].
%\end{abstract}

%% Erzeugung von Verzeichnissen %%%%%%%%%%%%%%%%%%%%%%%%%%%%%%%%%%%%%%%
\tableofcontents			% Inhaltsverzeichnis

%% Der Text %%%%%%%%%%%%%%%%%%%%%%%%%%%%%%%%%%%%%%%%%%%%%%%%%%%%%%%%%%%
\section{Erstelle die Startseite (Home) einer beliebigen Webseite(Titel, Text, Buttons, Links) unter Berücksichtigung der Gestaltgesetze und Farbwahl.}
\label{sec:AufgabeEins}
Beim geschriebenen HTML-Code handelt es sich um folgenden:

\lstinputlisting[language=HTML, caption=Webseite]{../index.html} \quelle{ich}
\cite[vgl.][2]{Hönemann2019:Farbwahrnehmung} \\
\cite[vgl.][2][10][30]{Hönemann2019:Farbwahrnehmung, Hönemann2019:Bewegungswahrnehmung, Schultz:Gestaltgesetze} \\
\cite{Hönemann2019:Bewegungswahrnehmung} \\
\cite{Schultz:Gestaltgesetze}\\
\cite{GoldsteinE.Bruce2015W:dG}\\
Die Html-Datei wird als extra Datei mit abgegeben.


\subsection{Drei Gestaltgesetze und eine harmonische Farbauswahl.}
\label{sec:AufgabeEinsA}

%\part[Erster Streich]{wenige Erläuterungen}
\section[Überschriften]{Überschrift und das Inhaltsverzeichnis}
Die meisten Artikel werden ab \verb#\section{}# gegliedert. Es steht jedoch grundsätzlich auch der sehr gewaltige, übergeordnete Gliederungspunkt \verb#\part{}# zur Verfügung.

Durch das optionale Argument [Kurzform] in \verb#\section[Kurzform]{Überschrift}# kann für das Inhaltsverzeichnis eine alternative Formulierung der Überschrift definiert werden. Das heißt, dass bei Verwendung des Arguments der Inhalt der eckigen Klammern in das Inhaltsverzeichnis aufgenommen wird, die Überschrift im Fließtext jedoch durch den Inhalt der geschweiften Klammern gebildet wird. 

Die Überschrift dieses Abschnitts ist ein Beispiel dafür: Erstellen Sie ein Beispiel!

\subsection{Erstellung des Beispiels}
Entfernen Sie das \%-Zeichen vor \verb#\tableofcontens#, speichern Sie das Dokument (Datei, Speichern unter) und compilieren Sie das Dokument (F7) zweimal. Sehen Sie sich das Beispiel an (F5). Beachten Sie die Bezeichnung im Inhaltsverzeichnis und im Fließtext.

\section{weitere Gliederungspunkte}
Die folgenden Gliederungsschritte nach \verb#\section[]{}# und \verb#\subsection[]{}# sind 
\begin{verbatim}
\subsubsection[]{},\paragraph[]{} und \subparagraph[]{}.
\end{verbatim}

Sie alle sind erreichbar über das Menü: Einfügen, Überschrift.

\subsection{die Sternvariante}
Durch den * wird der Gliederungspunkt weder nummeriert, noch als Kolumnentitel gesetzt und es erfolgt kein Eintrag
ins Inhaltsverzeichnis.

Ein Beispiel ist die nächste Subsection:

\subsection*{das Beispiel}
Diesen Gliederungspunkt findet man im Inhaltsverzeichnis nicht!
An dieser Stelle möchte ich Sie auf Besonderheiten der KOMA-Scriptklasse hinweisen, die Sie im scrguide nachlesen können:
\begin{verbatim}
\addpart[Kurzform]{Überschrift}
\addpart*{Überschrift}
\addchap[Kurzform]{Überschrift}
\addchap*{Überschrift}
\addsec[Kurzform]{Überschrift}
\addsec*{Überschrift}
und
\minisec{Überschrift}
\end{verbatim}
Es liefert einige interessante Optionen!\footnote{Hierin kann auch nachgelesen werden, wie individuelle Kopfzeilen mit scrpage2 erstellt werden können.}

\section{persönlicher Tipp}
Arbeiten Sie mit TeXnicCenter stets mit Projekten, speichern Sie die einzelnen Teile Ihrer Arbeit in einzelnen Dateien und binden Sie diese mit Hilfe von \verb#\input{Dateiname}# in das Hauptdokument ein.

Persönlich nutze ich einen eigene Datei für jede Section. 

Viel Spaß! Für Rückfragen, die diese Vorlage betreffen, stehe ich Ihnen gern in der Mailingliste von TXC zur Verfügung. Ansonsten sind die Dokumente \texttt{lshort}, \texttt{l2tabu}, die \texttt{FAQ der Newsgroup de.text.tex} und natürlich der \texttt{scrguide} immer sehr hilfreich.

%% Erzeugung von Verzeichnissen %%%%%%%%%%%%%%%%%%%%%%%%%%%%%%%%%%%%%%%
\newpage
\listoftables			% Tabellenverzeichnis
\newpage
\listoffigures				% Abbildungsverzeichnis
\newpage
\lstlistoflistings
\newpage
%% Bibliographie unter Verwendung von dinnat %%%%%%%%%%%%%%%%%%%%%%%%%%
%\setbibpreamble{Präambel}		% Text vor dem Verzeichnis
%\bibliographystyle{dinat}
%\bibliography{bibliographie}	% Sie benötigen einen *.bib-Datei
%\addcontentsline{toc}{chapter}{Quellenverzeichnis}
\nocite{*} % adds all entries in the bib file to the bibliography
%\printbibliography
%\printbibheading
%\renewcaptionname{ngerman}{\bibname}{Quellenverzeichnis}             %Umbennenen des Bibliography Titels
\newpage
\pagenumbering{Roman}
%\addcontentsline{toc}{section}{Quellenverzeichnis}
%\renewcommand{\refname}{Quellenverzeichnis}
%\renewcommand{\thebibliography}{Internetrecherche}
\renewcommand{\bibname}{Literatur}
\nocite{*} % adds all entries in the bib file to the bibliography

\printbibliography[nottype = online, title = Literaturverzeichnis,heading	= subbibliography]
\printbibliography[
	type 		= online,						% Put only references of the type "online" here.
	title		= {Online-Quellen},	%Setzt den Titel für den subbibliograpy			 
	heading	= subbibliography
]

\end{document}