%% Basierend auf einer TeXnicCenter-Vorlage von Tino Weinkauf.
%%%%%%%%%%%%%%%%%%%%%%%%%%%%%%%%%%%%%%%%%%%%%%%%%%%%%%%%%%%%%%

%%%%%%%%%%%%%%%%%%%%%%%%%%%%%%%%%%%%%%%%%%%%%%%%%%%%%%%%%%%%%
%% OPTIONEN
%%%%%%%%%%%%%%%%%%%%%%%%%%%%%%%%%%%%%%%%%%%%%%%%%%%%%%%%%%%%%
%%
%% ACHTUNG: Sie benötigen ein Hauptdokument, um diese Datei
%%          benutzen zu können. Verwenden Sie im Hauptdokument
%%          den Befehl "\input{dateiname}", um diese
%%          Datei einzubinden.
%%

%%%%%%%%%%%%%%%%%%%%%%%%%%%%%%%%%%%%%%%%%%%%%%%%%%%%%%%%%%%%%
%% OPTIONEN FÜR ABSTÄNDE
%%%%%%%%%%%%%%%%%%%%%%%%%%%%%%%%%%%%%%%%%%%%%%%%%%%%%%%%%%%%%

%%Abstand zwischen den Absätzen: halbe Höhe vom kleinen x
\setlength{\parskip}{0.5ex}

%%Einzug am Anfang eines Absatzes: auf Null setzen
%\setlength{\parindent}{0ex}

%%Zeilenabstand: 1.5 fach
%% ==> Erwägen Sie, anstelle dieses Kommandos das Paket 'setspace' zu verwenden.
%\linespread{1.5}

%%Abstände zu Seitenränder definieren
\usepackage[left=2.5cm, right=2.5cm, top=2.5cm, bottom=2.5cm]{geometry}

%%%%%%%%%%%%%%%%%%%%%%%%%%%%%%%%%%%%%%%%%%%%%%%%%%%%%%%%%%%%%
%% OPTIONEN FÜR Quellcode
%%%%%%%%%%%%%%%%%%%%%%%%%%%%%%%%%%%%%%%%%%%%%%%%%%%%%%%%%%%%%
%%Um Quellcode einzubinden wird folgender package benötigt:
%\usepackage{listings}      % Quellcode einbinden

%% Um zusätzlich den Quellcode farblich markieren/kennzeichnen zu wollen, muss folgendes Packages definiert werden:
%\usepackage{xcolor}         % Text farbig markieren                    

\definecolor{hellgelb}{rgb}{1,1,0.8}
\definecolor{colKeys}{rgb}{0,0,1}
\definecolor{colIdentifier}{rgb}{1,0,0}
\definecolor{colComments}{rgb}{0,0.7,0.4}
\definecolor{colString}{rgb}{0,0.5,0}

\lstset{%
    float=hbp,%
    basicstyle=\ttfamily\small, %
    identifierstyle=\color{colIdentifier}, %
    keywordstyle=\color{colKeys}, %
    stringstyle=\color{colString}, %
    commentstyle=\color{colComments}, %
    columns=flexible, %
    tabsize=2, %
    frame=single, %
    extendedchars=true, %
    showspaces=false, %
    showstringspaces=false, %
    numbers=left, %
    numberstyle=\tiny, %
    breaklines=true, %
    backgroundcolor=\color{hellgelb}, %
    breakautoindent=true, %
    captionpos=b%
}

\lstdefinelanguage{CSS}
{morekeywords={color,background,margin,padding,font,weight,display,position,top,left,right,bottom,list,style,border,size,white,space,min,width},
sensitive=false,
morecomment=[l]{//},
morecomment=[s]{/*}{*/},
morestring=[b]",
}
%%%%%%%%%%%%%%%%%%%%%%%%%%%%%%%%%%%%%%%%%%%%%%%%%%%%%%%%%%%%%
%% OPTIONEN FÜR KOPF- UND FUSSZEILEN
%%%%%%%%%%%%%%%%%%%%%%%%%%%%%%%%%%%%%%%%%%%%%%%%%%%%%%%%%%%%%
%%Beispiel für recht nette Kopf- und Fußzeilen
%% ==> Nutzen Sie '\usepackage{fancyhdr}' und '\pagestyle{fancy}'
%% ==> im Hauptdokument, um diese zu benutzen.
%\pagestyle{fancy}
%\renewcommand{\chaptermark}[1]{\markboth{#1}{}}
%\renewcommand{\sectionmark}[1]{\markright{\thesection\ #1}}
%\fancyhf{}
%\fancyhead[LE,RO]{\thepage}
%\fancyhead[LO]{\rightmark}
%\fancyhead[RE]{\leftmark}
%\fancypagestyle{plain}{%
%    \fancyhead{}
%    \renewcommand{\headrulewidth}{0pt}
%}


